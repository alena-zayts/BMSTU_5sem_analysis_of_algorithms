\chapter{Аналитическая часть}

В данном разделе будут рассмотрены основные идеи трех алгоритмов сортировок - сортировки вставками, сортировки пузырьком и шейкерной сортировки.

\section{Сортировка вставками}

 Сортировка вставками — достаточно простой алгоритм. Сортируемый массив можно разделить на две части — отсортированную и неотсортированную. В начале сортировки первый элемент массива считается отсортированным, все остальные — не отсортированные. Начиная со второго элемента массива и заканчивая последним, алгоритм вставляет очередной элемент из неотсортированной части массива в нужную позицию в отсортированной части. 
 
 Таким образом, за один шаг сортировки отсортированная часть массива увеличивается на один элемент, а неотсортированная часть массива уменьшается на один элемент. Данный процесс вставки продолжается до тех пор, пока все элементы исходного списка не окажутся в расширяющейся отсортированной части списка \cite{article_insert}.


\section{Сортировка пузырьком}

Сортировка пузырьком — один из самых известных алгоритмов сортировки. Алгоритм  состоит  из  повторяющихся  проходов  по  сортируемому массиву. За каждый проход элементы последовательно сравниваются попарно, и, если порядок в паре неверный, выполняется обмен элементов. Проходы  по  массиву повторяются  N-1 раз.  При каждом  проходе  алгоритма  по  внутреннему  циклу  очередной  наибольший  элемент  массива ставится на своё место в конце массива рядом с предыдущим «наибольшим элементом», а наименьший элемент перемещается на одну позицию к  началу  массива  («всплывает»  до  нужной  позиции,  как  пузырёк  в воде — отсюда  и  название алгоритма) \cite{first_book}.

\section{Шейкерная сортировка}

Алгоритм шейкерной сортировки  является  модификацией  пузырьковой  сортировки  по направлению движения.  Отличия от нее заключаются в том, что при прохождении части массива, происходит проверка, были ли перестановки. Если их не было, значит эта часть массива уже упорядочена, и она исключается из дальнейшей обработки. Кроме того, при прохождении массива от начала к концу минимальные элементы перемещаются в самое начало, а максимальный элемент сдвигается к концу массива \cite{shaker_sort}.


\section*{Вывод}
В данном разделе были рассмотрены идеи, лежащие в основе рассматриваемых алгоритмов сортировок - сортировки вставками, сортировки пузырьком и шейкерной сортировки.