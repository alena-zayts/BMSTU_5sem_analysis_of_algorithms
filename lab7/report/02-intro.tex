\chapter*{Введение}
\addcontentsline{toc}{chapter}{Введение}

Словарь (англ. dictionary, map) — абстрактный тип данных, позволяющий хранить набор значений, обращение к которым происходит по ключам. Ключи должны допускать сравнение друг с другом. Примеры словарей достаточно разнообразны. Например, обычный толковый словарь хранит определения слов (являющиеся значениями), сопоставленные с самими словами (являющимися ключами), а банковская база данных может хранить данные клиентов, сопоставленные с номерами счетов.

Одной из основных операций в словаре является поиск значения по ключу. Словари могут содержать большое количество элементов, поэтому вопрос скорости поиска в них является важным~\cite{first_article}. 


Целью данной работы является разработка эффективного алгоритма поиска по словаре.


В рамках выполнения работы необходимо решить следующие задачи: 
\begin{enumerate}[label={\arabic*)}]
	\item изучить три алгоритма поиска в словаре: алгоритм полного перебора, алгоритм бинарного поиска и алгоритм поиска в сегментированном словаре;
	\item разработать и реализовать изученные алгоритмы;
	\item провести сравнительный анализ трудоёмкости реализаций алгоритмов на основе теоретических расчетов (в среднем; в лучшем и худшем случаях);
	\item провести сравнительный анализ процессорного времени выполнения реализаций алгоритмов на основе экспериментальных данных;
	\item провести сравнительный анализ количества сравнений с заданным ключом, необходимых для поиска каждым алгоритмом значения по ключу и для определения отсутствия заданного ключа в словаре.
\end{enumerate}
