\chapter{Аналитическая часть}

В данном разделе будет приведена теория, необходимая для разработки и реализации двух алгоритмов решения задачи коммивояжера: алгоритма полного перебора и муравьиного алгоритма.

\section{Задача коммивояжера}

В задаче коммивояжера рассматривается n городов  и матрица попарных расстояний между ними. Требуется найти такой порядок  посещения  городов,  чтобы  суммарное  пройденное  расстояние было минимальным, каждый город посещался ровно один раз и  коммивояжер  вернулся  в  тот  город,  с  которого  начал  свой  маршрут.  Другими  словами,  во  взвешенном  полном  графе  требуется найти гамильтонов цикл минимального веса ~\cite{second_article}. 

\section{Алгоритм полного перебора для решения задачи коммивояжера}

Cуть алгоритма полного перебора для решения задачи коммивояжера заключается в переборе всех вариантов путей и нахождении кратчайшего. Преимуществом является его результат - точное решение, недостатком - длительность вычислений при относительно небольшом пространстве поиска ~\cite{third_article}. Эта задача является NP-трудной, и точный переборный алгоритм ее решения имеет факториальную сложность ~\cite{first_article}.


\section{Муравьиный алгоритм для решения задачи коммивояжера}

Моделирование поведения муравьев связано с распределением феромона на тропе — ребре графа в задаче коммивояжера. При этом вероятность включения ребра в маршрут отдельного муравья пропорциональна количеству феромона на этом ребре, а количество откладываемого феромона пропорционально длине маршрута. Чем короче маршрут, тем больше феромона будет отложено на его ребрах. При этом избежать преждевременной сходимости можно, моделируя отрицательную обратную связь в виде испарения феромона. С учетом особенностей задачи коммивояжера, мы можем описать локальные правила поведения муравьев при выборе пути.

\begin{itemize}
	\item Муравьи имеют собственную «память» в виде списка уже посещенных городов. Обозначим через $J_{i, k}$ список городов, которые необходимо посетить муравью k, находящемуся в городе i.

	\item Муравьи обладают «зрением» — видимость есть эвристическое желание посетить город j, если муравей находится в городе i. Будем считать, что видимость обратно пропорциональна расстоянию между городами i и j — $D_{ij}$: $\eta_{ij} = \frac{1}{D_{ij}}$.
	\item Муравьи обладают «обонянием» — они могут улавливать след феромона, подтверждающий желание посетить город j из города i, на основании опыта других муравьев. Количество феромона на ребре (i, j) в момент времени t обозначим через $\tau_{ij}(t)$.
\end{itemize}

На этом основании мы можем сформулировать вероятностно-пропорциональное правило \ref{form:way}, определяющее вероятность перехода k-ого муравья из города i в город j:


\begin{equation}
	\label{form:way}
	P_{ij,k}(t) = \begin{cases}
		\frac {(\tau_{ij}(t))^{\alpha }(\eta_{ij}(t))^{\beta }}{\sum\limits_{l \in J_{i, k}}(\tau_{il}(t))^{\alpha }(\eta_{il}(t))^{\beta }}, \textrm{$j \in J_{i, k}$,} \\
		0, \textrm{иначе}
	\end{cases},
\end{equation}
где $\alpha - $ параметр влияния длины пути, $\beta - $ параметр влияния феромона. При $\alpha = 0$ будет выбран ближайший город, что соответствует жадному алгоритму в классической теории оптимизации. Если $\beta=0$ работает лишь феромонное усиление, что влечет за собой быстрое вырождение маршрутов к одному субоптимальному решению \cite{second_article}.

Пройдя ребро (i, j), муравей откладывает на нем некоторое количество феромона, которое должно быть связано с оптимальностью сделанного выбора.
Пусть $T_{k}(t)$ есть маршрут, пройденный муравьем k к моменту времени t , а $L_{k}(t)$ — длина этого маршрута. Пусть также Q — параметр, имеющий значение порядка длины оптимального пути. Тогда откладываемое количество феромона может быть задано в виде \ref{form:fer}:

\begin{equation}
	\label{form:fer}
	\Delta \tau_{ij, k}(t) = \begin{cases}
		\frac{Q}{L_{k}(t)}, \textrm{$(i, j) \in T_{k}(t)$,} \\
		0, \textrm{иначе.}
	\end{cases}
\end{equation}

Правила внешней среды определяют, в первую очередь, испарение феромона. Пусть $p \in [0, 1]$ есть коэффициент испарения, тогда правило испарения имеет вид
\ref{form:isp}:

\begin{equation}
	\label{form:isp}
	\tau_{ij}(t+1) = (1-p)\tau_{ij}(t) + \Delta \tau_{ij}(t), \Delta \tau_{ij} = \sum_{k=1}^m \Delta \tau_{ij, k}(t),
\end{equation}
где m — количество муравьев в колонии.

В начале алгоритма количество феромона на ребрах принимается равным небольшому положительному числу. При этом необходимо следить, чтобы количество феромона на существующем ребре не обнулилось в ходе испарения. Общее количество муравьев остается постоянным и равным количеству городов, каждый муравей начинает маршрут из своего города. 


\section{Вывод из аналитической части}
Были рассмотрены идеи и материалы, необходимые для  разработки и реализации двух алгоритмов решения задачи коммивояжера: алгоритма полного перебора и муравьиного алгоритма.

