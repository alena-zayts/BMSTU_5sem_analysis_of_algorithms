\chapter{Технологическая часть}

В данном разделе производится выбор средств реализации, а также приводятся требования к программному обеспечению (ПО), листинги реализованных алгоритмов и тесты для программы.

\section{Требования к ПО}

На вход программе подаются две строки (регистрозависимые), а на выходе должно быть получено искомое расстояние, посчитанное с помощью каждого реализованного алгоритма: для расстояния Левенштейна - итерационный и рекурсивный (с кешем и без), а для расстояния Дамерау-Левенштейна - рекурсивный без кеша. Также необходимо вывести затраченное каждым алгоритмом процессорное время и пиковый объем выделенной памяти.

\section{Средства реализации}

В качестве языка программирования для реализации данной лабораторной работы был выбран язык Python  \cite{PythonBook}. Он позволяет быстро реализовывать различные алгоритмы без выделения большого времени на проектирование сруктуры программы и выбор типов данных. 

Кроме того, в Python есть библиотека time, которая предоставляет функцию process\_time для замра процессорного времени \cite{process_time_text}, а также библиотека memory\_profiler, предоставляющая функцию memory\_usage, которая позволяет замерить пиковый объем памяти, выделенный при работе функции \cite{memory_usage_text}.

В качестве среды разработки выбран PyCharm. Он является кросс-платформенным, а также предоставляет удобный и функциональнаый отладчик и средства для рефакторинга кода, что позволяет быстро находить и исправлять ошибки \cite{pycharm}.

\section{Листинг кода}

В листингах \ref{lev_matrix} - \ref{dlev_recursion} представлены реализации рассматриваемых алгоритмов.

\begin{lstlisting}[caption=Функция поиска расстояния Левенштейна с заполнением матрицы расстояний,
	label={lev_matrix}]
	def lowenstein_dist_matrix_classic(str1, str2):
		# +1 because of an empty string
		n = len(str1) + 1
		m = len(str2) + 1
			matrix = [[0 for i in range(m)] for j in range(n)]  # MATCH
		
		# fill with trivial rules
		for i in range(1, n):
			matrix[i][0] = i  # DELETION
		for j in range(1, m):
			matrix[0][j] = j  # INSERTION
		
		# fill the rest of the matrix
		for i in range(1, n):
			for j in range(1, m):
				insertion = matrix[i][j - 1] + 1
				deletion = matrix[i - 1][j] + 1
				replacement = matrix[i - 1][j - 1] + int(str1[i - 1] != str2[j - 1])
				
				matrix[i][j] = min(insertion, deletion, replacement)
		
		
		return matrix[n - 1][m - 1]
\end{lstlisting}


\begin{lstlisting}[caption=Функция рекурсивного алгоритма поиска расстояния Левенштейна без кеширования,
	label={lev_recursion_classic}]
	def lowenstein_dist_recursion_classic(str1, str2):
		# trivial rules
		if not str1:
			return len(str2)
		elif not str2:
			return len(str1)
		
		insertion = lowenstein_dist_recursion_classic(str1, str2[:-1]) + 1
		deletion = lowenstein_dist_recursion_classic(str1[:-1], str2) + 1
		replacement = lowenstein_dist_recursion_classic(str1[:-1], str2[:-1]) + int(str1[-1] != str2[-1])
		
		return min(insertion, deletion, replacement)
\end{lstlisting}

\begin{lstlisting}[caption=Функция рекурсивного алгоритма поиска расстояния Левенштейна c кешированием,
	label={lev_recursion_optimized}]
	def lowenstein_dist_recursion_optimized(str1, str2):
		def _lowenstein_dist_recursion_optimized(str1, str2, matrix):
			len1 = len(str1)
			len2 = len(str2)
			
			# trivial rules
			if not len1:
				matrix[len1][len2] = len2
			elif not len2:
				matrix[len1][len2] = len1
			else:
				# insertion
				if matrix[len1][len2 - 1] == -1:
					_lowenstein_dist_recursion_optimized(str1, str2[:-1], matrix)
				# deletion
				if matrix[len1 - 1][len2] == -1:
					_lowenstein_dist_recursion_optimized(str1[:-1], str2, matrix)
				# replacement
				if matrix[len1 - 1][len2 - 1] == -1:
					_lowenstein_dist_recursion_optimized(str1[:-1], str2[:-1], matrix)
				
				matrix[len1][len2] = min(matrix[len1][len2 - 1] + 1,
				matrix[len1 - 1][len2] + 1,
				matrix[len1 - 1][len2 - 1] + int(str1[-1] != str2[-1]))
		
		# +1 because of an empty string
		n = len(str1) + 1
		m = len(str2) + 1
		matrix = [[-1 for i in range(m)] for j in range(n)]
		_lowenstein_dist_recursion_optimized(str1, str2, matrix)
		
		return matrix[n - 1][m - 1]
\end{lstlisting}

\begin{lstlisting}[caption=Функция рекурсивного алгоритма поиска расстояния Дамерау-Левенштейна,
	label={dlev_recursion}]
	def damerau_lowenstein_dist_recursion(str1, str2):
		# trivial rules
		if not str1:
			return len(str2)
		elif not str2:
			return len(str1)
		
		insertion = lowenstein_dist_recursion_classic(str1, str2[:-1]) + 1
		deletion = lowenstein_dist_recursion_classic(str1[:-1], str2) + 1
		replacement = lowenstein_dist_recursion_classic(str1[:-1], str2[:-1]) + int(str1[-1] != str2[-1])
		
		if len(str1) > 1 and len(str2) > 1 and str1[-1] == str2[-2] and str1[-2] == str2[-1]:
			xchange = lowenstein_dist_recursion_classic(str1[:-2], str2[:-2]) + int(str1[-1] != str2[-1])
			return min(insertion, deletion, replacement, xchange)
		else:
			return min(insertion, deletion, replacement)
\end{lstlisting}

\section{Тестирование}

В таблице \ref{tabular:test} приведены функциональные тесты для алгоритмов вычисления расстояния Левенштейна и Дамерау — Левенштейна. Тесты пройдены успешно.

\begin{table}[h]
	\begin{center}
		\caption{\label{test} Тесты}
		\begin{tabular}{|c|c|c|c|}
			\hline
			&                    & \multicolumn{2}{c|}{\bfseries Ожидаемый результат}    \\ \cline{3-4}\hline
			Строка 1& Строка 2 & Алг. Левенштейна & Алг. Дамерау-Левенштейна \\ [0.5ex] 
			\hline
			 &  & 0 & 0\\
			\hline
			abc & abc & 0 & 0\\
			\hline
			ab & a & 1 & 1\\
			\hline
			a & ab & 1 & 1\\
			\hline
			see & sea & 1 & 1\\
			\hline
			1234 & 1324 & 2 & 1\\
			\hline
			hello & ehlla & 3 & 2\\
			\hline
			cat & pop & 3 & 3\\
			\hline
			кот & скат & 2 & 2\\
			\hline
		\end{tabular}
	\end{center}
\end{table}


\section*{Вывод}

Был производен выбор средств реализации, реализованы и протестированы алгоритмы поиска расстояний: Левенштейна - итерационный и рекурсивный (с кешем и без), Дамерау-Левенштейна - рекурсивный без кеша
