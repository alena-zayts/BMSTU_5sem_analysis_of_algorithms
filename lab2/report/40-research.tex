\chapter{Исследовательская часть}

\section{Пример работы}

На рисунке \ref{img:work_example} приведен пример работы программы.

\boximg{160mm}{work_example}{Пример работы программы}

\section{Технические характеристики}

Технические характеристики устройства, на котором выполнялось тестирование:

\begin{itemize}
	\item операционная система: Windows 10;
	\item оперативная память: 16 Гб;
	\item процессор: Intel® Core™ i5-8259U.
\end{itemize}

Во время тестирования ноутбук был включен в сеть питания и нагружен только встроенными приложениями окружения и системой тестирования.

\section{Время выполнения реализаций алгоритмов}

 Все реализации алгоритмов сравнивались на случайно сгенерированных квадратных матрицах размерностями n*n, где n изменялось от 100 до 1000 с шагом 100. Так как замеры времени имеют некоторую погрешность, они для каждой размерности и каждой реализации алгоритма производились 10 раз, а затем вычислялось среднее время работы реализации с матрицами.
 
На рисунке \ref{img:time_all} приведены результаты сравнения времени работы всех реализаций. Как видно на графике, теоретические расчеты подтверждаются: все алгоритмы кубически зависят от размерностей матриц, при этом алгоритм Винограда работает дольше всех, а оптимизированный алгоритм Винограда - меньше всех.

\img{120mm}{time_all}{Сравнение времени работы реализаций алгоритмов}



\section*{Вывод}

Были подтверждены теоретические расчеты: алгоритм Винограда работает дольше стандартного, а оптимизированный алгоритм Винограда - меньше.
 
Таким образом, несмотря на сложность алгоритма Винограда по сравнению со стандартным, меньшая доля умножений в нем при применении оптимизаций позволяет получить меньшую трудоемкость

