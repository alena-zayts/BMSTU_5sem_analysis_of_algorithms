\chapter*{Введение}
\addcontentsline{toc}{chapter}{Введение}

Целью данной лабораторной работы является получение навыков оценки трудоемкости алгоритмов на примере трех алгоритмов сортировок.

Под сортировкой понимается упорядочивание элементов по какому-либо признаку. Сортировка  относится  к  важнейшему  классу  алгоритмов  обработки данных и осуществляется большим количеством способов ~\cite{first_book}.

Алгоритмы сортировки имеют большое практическое применение и часто встречаю там, где речь идет об обработке и хранении больших объемов информации.  Некоторые задачи обработки данных решаются проще, если данные заранее упорядочить. Упорядоченные объекты содержатся в телефонных книгах, ведомостях налогов, в библиотеках, в оглавлениях, в словарях ~\cite{first_article}.


В настоящее время, в связи с постоянно растущими объемами данных, вопрос эффективности сортировки данных не теряет свою актуальность.



В рамках выполнения работы необходимо решить следующие задачи: 
\begin{enumerate}[label={\arabic*)}]
	\item изучить три алгоритма сортировок: сортировку вставками, сортировку пузырьком и шейкерную сортировку;
	\item разработать и реализовать изученные алгоритмы;
	\item провести сравнительный анализ трудоёмкости реализаций алгоритмов на основе теоретических расчетов;
	\item провести сравнительный анализ процессорного времени выполнения реализаций алгоритмов на основе экспериментальных данных.
\end{enumerate}
