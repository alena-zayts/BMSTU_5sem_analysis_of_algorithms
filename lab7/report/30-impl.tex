\chapter{Технологическая часть}

В данном разделе производится выбор средств реализации, приводятся требования к ПО, листинги реализованных алгоритмов поиска в словаре, а также результаты их тестированиия.

\section{Требования к ПО}

На вход программе подается ключ, на выходе ожидается значение, найденное в словаре по заданному ключу каждым реализованным алгоритмом, или значение -1, если такого ключа в словаре нет. 

\section{Выбор средств реализации}

В качестве языка программирования для реализации данной лабораторной работы был выбран язык Python  \cite{PythonBook}. Он позволяет быстро реализовывать различные алгоритмы без выделения большого времени на проектирование сруктуры программы и выбор типов данных. 

Кроме того, в Python есть библиотека time, которая предоставляет функцию process\_time для замера процессорного времени \cite{process_time_text}.

В качестве среды разработки выбран PyCharm. Он является кросс-платформенным, а также предоставляет удобный и функциональнаый отладчик и средства для рефакторинга кода, что позволяет быстро находить и исправлять ошибки \cite{pycharm}.

\section{Листинги кода}

В листинге \ref{full} представлена реализация алгоритма полного перебора поиска в словаре.

\clearpage
\begin{lstlisting}[caption=Алгоритм полного перебора,
	label={full}]
def full_search(my_dict, key):
	for cur_key in my_dict.keys():
		if cur_key == key:
			return my_dict[cur_key]
	return -1
\end{lstlisting}

В листинге \ref{bin} представлена реализация алгоритма бинарного поиска в словаре.


\begin{lstlisting}[caption=Алгоритм бинарного поиска в словаре,
	label={bin}]
def binary_search(my_dict, key):
	my_keys = list(my_dict.keys())
	my_len = len(my_keys)
	l = 0
	r = my_len - 1
	
	while l <= r:
		middle = (r + l) // 2
		cur_key = my_keys[middle]
		
		if cur_key == key:
			return my_dict[cur_key]
		elif cur_key > key:
			r = middle - 1
		else:
			l = middle + 1
	
	return -1
\end{lstlisting}

\clearpage
В листинге \ref{help} представлена реализация функции для создания сегментированного словаря.


\begin{lstlisting}[caption=Функция для создания сегментированного словаря,
	label={help}]

def segmentate(my_dict):
	seg_frequency_dict = {letter: 0 for letter in "abcdefghijklmnopqrstuvwxyz"}
	for key in my_dict.keys():
		seg_frequency_dict[key[0]] += 1
	
	seg_frequency_dict = sort_by_values(seg_frequency_dict, reverse=True)
	
	new_dict = {letter: dict() for letter in seg_frequency_dict.keys()}
	for key in my_dict.keys():
		new_dict[key[0]].update({key: my_dict[key]})
	for key in new_dict:
		new_dict[key] = sort_by_keys(new_dict[key])
	
	return new_dict

\end{lstlisting}

В листинге \ref{seg} представлена реализация алгоритма поиска в сегментированном словаре.


\begin{lstlisting}[caption=Алгоритм поиска в сегментированном словаре,
	label={seg}]
def segment_search(my_dict, key):
	for k in my_dict:
		if key[0] == k:
			if my_dict[k]:
				return binary_search(my_dict[k], key)
			else:
				return -1
	return -1
\end{lstlisting}

\clearpage
\section{Тестирование}

В таблице ~\ref{tab:test_rec} приведены функциональные тесты для алгоритмов поиска в словаре. Все тесты пройдены успешно этим каждым алгоритмом.

\begin{table}[h!]
	\begin{center}
		\caption{\label{tab:test_rec} Функциональные тесты}
		\begin{tabular}{|c | c | c | c |}
			\hline
			Ключ & Словарь & Ожидание & Результат \\
			\hline
			"few" & \texttt{\{\}} & -1 & -1 \\
			"few" & \texttt{\{"few": "неск."\,, "little": "немн." \}} & "неск." & "неск." \\
			"much" & \texttt{\{"few": "неск."\,, "little": "немн." \}} & -1 & -1 \\
			\hline
		\end{tabular}
	\end{center}
	\end{table}



\section{Вывод из технологического раздела}

Был производен выбор средств реализации, приведены требования к ПО, реализованы и протестированы алгоритмы поиска в словаре.
