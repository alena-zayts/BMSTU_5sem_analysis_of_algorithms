\chapter*{Введение}
\addcontentsline{toc}{chapter}{Введение}

Существуют различные способы написания программ, одним из которых является использование параллельных вычислений. Параллельные вычисления – способ организации компьютерных вычислений, при котором программы разрабатываются как набор взаимодействующих вычислительных процессов, работающих параллельно (одновременно) ~\cite{first_article}. 

Основная цель параллельных вычислений – уменьшение времени
решения задачи. Многие необходимые для нужд практики задачи требуется решать в реальном времени или для их решения требуется очень большой объем вычислений. Таким трудоемким алгоритмом является, например, алгоритм трассировки лучей - один из методов построения реалистичных сцен c учетом теней, эффектов отражения, преломленияи т.д..

Целью данной работы является изучение организации параллельных вычислений на базе алгоритма трассировки лучей.


В рамках выполнения работы необходимо решить следующие задачи: 
\begin{enumerate}[label={\arabic*)}]
	\item изучить основы параллельных вычислений;
	\item изучить алгоритм трассировки лучей;
	\item разработать алгоритм трассировки лучей;
	\item реализовать последовательную и параллельную трассировку лучей;
	\item провести сравнительный анализ времени работы реализаций.
\end{enumerate}
