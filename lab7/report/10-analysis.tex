\chapter{Аналитическая часть}

В данном разделе будет приведена теория, необходимая для разработки и реализации трех алгоритмов поиска в словаре: алгоритма полного перебора, алгоритма бинарного поиска и алгоритма поиска в сегментированном слолваре. 

\section{Алгоритм полного перебора}

Алгоритм полного перебора сводится к последовательному прохождению по всем ключам словаря и их сравнению с заданным ключом. 


Этот алгоритм считается методом <<грубой силы>>~\cite{kekw}. Зато он может производиться в неотсортированном словаре, а добавление новых элементов в такой словарь не вызывает затруднений -- их можно добавлять на любую позицию. Дополнительных затрат по памяти также не требуется.

\section{Алгоритм бинарного поиска}

Двоичный поиск (бинарный поиск)~\cite{second_article} — алгоритм поиска объекта (в нашем случае - значения) по заданному признаку (ключу) в множестве объектов, упорядоченных по тому же самому признаку (по ключу), принцип работы которого заключается в том, что на каждом шаге множество объектов делится на две части и в работе остаётся та часть множества, где находится искомый объект. 

Данный алгоритм работает за логарифмическое время, дополнительных затрат по памяти не требует. Однако алгоритм бинарного поиска подразумевает, что массив ключей отсортирован. Это вызывает трудности при добавлении новых элементов в словарь, особенно если словарь хранится в постоянной памяти (например, на диске).

Алгоритм вставки элемента в упорядоченный массив заключается в последовательном анализе элементов массива, начиная с последнего и, при необходимости, сдвиге анализируемого элемента вправо на одну позицию, для того, чтобы освободить место для вставляемого элемента. Сдвиги проводятся до тех пор, пока не будет найдено место для вставляемого элемента, соответствующее его значению ~\cite{third_article}.

\section{Алгоритм поиска в сегментированном словаре}

Словарь можно разбить на сегменты так, чтобы все элементы с некоторым общим признаком попадали в один сегмент. Признаком для строк может быть первая буква, для чисел -- остаток от деления. Элементы внутри одного сегмента можно также разбивать на подсегменты и так далее.

Затем сегменты упорядочиваются по значению частотной характеристики так, чтобы к элементам с наибольшей частотной характеристикой был самый быстрый доступ (в случае поиска полным перебором они должны быть ближе к началу). Такой характеристикой может послужить, например, размер сегмента. 

Вероятность обращения к определенному сегменту равна сумме вероятностей обращений к его ключам, то есть $P_i = \sum_{j}p_j$, где $P_i$ - вероятность обращения к $i$-ому сегменту, $p_j$ - вероятность обращения к $j$-ому элементу, который принадлежит $i$-ому сегменту. Если обращения ко всем ключам равновероятны, то можно заменить сумму на произведение: $P_i = N \cdot p$, где $N$ - количество элементов в $i$-ом сегменте, а $p$ - вероятность обращения к произвольному ключу.

Таким образом, для поиска ключа в словаре необходимо найти сегмент, которму принадлежит этот ключ (например, полным перебором), а затем осуществить поиск ключа в пределах найденного сегмента. Чтобы поиск в пределах сегмента осуществлялся эффективнее, можно упорядочить ключи в каждом сегменте и реализовать бинарный поиск. 

Сегментированный словарь позволяет уменьшить количество сравнений, необходимых для поиска в словаре. Однако он требует предобработки произвольного словаря для превращения его в сегментированный, а также дополнительной памяти под такую организацию. Объем дополнительной памяти определяется количеством выделяемых сегментов и подсегментов.

Также из-за необходимости поддержания структуры словаря усложняется операция добавления в словарь новых элементов. Необходимо определить, к какому сегменту относится новый ключ, если сегмент разбит на подсегменты -- соответсвующий подсегмент, а затем найти позицию внутри сегмента (подсегмента), в которую этот ключ необходимо вставить для сохранения упорядоченности. Также после добавления нового ключа сегмент (подсегмент) становится больше, поэтому необходимо проверить упорядоченность сегментов (подсегментов) по частотной характеристике и пересортировать их в случае необходимости. 

\section{Вывод из аналитической части}
Были рассмотрены идеи и материалы, необходимые для  разработки и реализации трех алгоритмов поиска в словаре: алгоритма полного перебора, алгоритма бинарного поиска и алгоритма поиска в сегментированном словаре. 

