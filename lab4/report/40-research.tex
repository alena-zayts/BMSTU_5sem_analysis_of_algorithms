\chapter{Исследовательская часть}

\section{Технические характеристики}

Технические характеристики устройства, на котором выполнялось тестирование:

\begin{itemize}
	\item операционная система: Windows 10;
	\item оперативная память: 16 Гб;
	\item процессор: Intel® Core™ i5-8259U;
	\item количество ядер: 4;
	\item количество логических процессоров: 8.
\end{itemize}

Во время тестирования ноутбук был включен в сеть питания и нагружен только встроенными приложениями окружения и системой тестирования.


\section{Сравнение времени выполнения реализаций алгоритмов}

Сравнивалось время работы (обычное, по таймеру) последовательной и параллельной реализаций алгоритма обратной трассировки лучей, причем во втором случае сравнивалось также время работы реализации в зависимости от количества потоков (1, 2, 4, ..., 4*количесттво логических ядер=32).

Перечисленные реализации сравнивались по времени обработки сцены в зависимости от количества объектов (прямоугольных параллелепипедов) в ней: от 5 до 35 с шагом 5. 
 
Так как некоторые реализации выполняются достаточно быстро, а замеры времени имеют некоторую погрешность, они для каждой реализации и каждого количества элементов на сцене выполнялись 10 раз, а затем вычислялось среднее время работы.
 

На рисунке \ref{img:time_all} приведены результаты сравнения времени работы всех реализаций на всех данных (в легенде количество потоков указано как n\_threads). 

\img{120mm}{time_all}{Сравнение времени работы реализаций в зависимости от количества элементов на сцене}

Последовательная реализация и параллельная реализация с одним потоком, как и ожидалось, работают примерно одинаковое количество времени, хотя при этом вторая немного дольше в связи с накладными расходами на создание потока. Эти две реализации затратили наибольшее количество времени из всех сравниваемых.

Далее с ростом числа потоков время работы соответствующей параллельной реализации уменьшается, так как независимые вычисления производятся одновременно на разных ядрах. Это происходит вплоть до момента, когда используются 8 потоков, то есть их количество равно количеству логических ядер в компьютере.

При дальнейшем увеличении числа потоков время работы параллельной реализации больше, чем в описанной выше наилучшей точке, так как количество потоков становится больше количества логических ядер в компьютере, и, соответственно, некоторые из них вынуждены ожидать освобождения занятого другим потоком процессора, который смог бы провести необходимые вычисления. В результате теряется смысл в выделении этих потоков, так как одновременной обработки каждого из них не происходит, а дополнительное время на их организацию затрачивается.



\section{Вывод из исследовательской часть}

Таким образом, далеко не всегда двукратное увеличение числа потоков улучшает результат по времени. Это происходит, пока количество потоков меньше или равно количеству логических ядер в ЭВМ. Поэтому рекомендуемым числом потоков можно назвать число, равное количеству логических процессоров.