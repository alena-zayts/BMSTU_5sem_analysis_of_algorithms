\chapter{Аналитическая часть}

\section{Стандартный алгоритм умножения матриц}

Пусть даны две прямоугольные матрицы $A[M \times N]$ и $B[N \times Q]$:
$$
A = 
\begin{bmatrix} 
	a_{11} & a_{12} & \cdots & a_{1n} \\
	a_{21} & a_{22} & \cdots & a_{2n} \\ 
	\vdots & \vdots & \ddots & \vdots \\ 
	a_{m1} & a_{m2} & \cdots & a_{mn}
\end{bmatrix}
B =   
\begin{bmatrix} 
	b_{11} & b_{12} & \cdots & b_{1q} \\
	b_{21} & b_{22} & \cdots & b_{2q} \\ 
	\vdots & \vdots & \ddots & \vdots \\ 
	b_{n1} & b_{n2} & \cdots & b_{nq}
\end{bmatrix}
$$
Тогда матрица $C[M \times Q]$ -- произведение матриц:
$$
C = 
\begin{bmatrix} 
	c_{11} & c_{12} & \cdots & c_{1q} \\
	c_{21} & c_{22} & \cdots & c_{2q} \\ 
	\vdots & \vdots & \ddots & \vdots \\ 
	c_{m1} & c_{m2} & \cdots & c_{mq}
\end{bmatrix},
$$
в которой каждый элемент вычисляется по формуле 1: 
$$c_{ij} = \sum_{k=1}^n a_{ik}b_{kj}, ~(i=1, 2, \ldots l;j=1, 2, \ldots n)~~(1)$$

Стандартный алгоритм действует именно по этой формуле.


\section{Алгоритм Винограда умножения матриц}

Если посмотреть на результат умножения двух матриц, то видно, что каждый элемент в нем представляет собой скалярное произведение соответствующих строки и столбца исходных матриц.
Можно заметить также, что такое умножение допускает предварительную обработку, позволяющую часть работы выполнить заранее.

Рассмотрим два вектора $V = (v_1, v_2, v_3, v_4)$ и $W = (w_1, w_2, w_3, w_4)$.
Их скалярное произведение равно: $V \cdot W = v_1w_1 + v_2w_2 + v_3w_3 + v_4w_4$, что эквивалентно (\ref{for:new}):
\begin{equation}
	\label{for:new}
	V \cdot W = (v_1 + w_2)(v_2 + w_1) + (v_3 + w_4)(v_4 + w_3) - v_1v_2 - v_3v_4 - w_1w_2 - w_3w_4.
\end{equation}

Кажется, что второе выражение задает больше работы, чем первое: вместо четырех умножений мы насчитываем их шесть, а вместо трех сложений - десять. Менее очевидно, что выражение в правой части последнего равенства допускает предварительную обработку: его части можно вычислить заранее и запомнить для каждой строки первой матрицы и для каждого столбца второй. На практике это означает, что над предварительно обработанными элементами нам придется выполнять лишь первые два умножения и последующие пять сложений, а также дополнительно два сложения \cite{vino}.

В конце нужно проверить кратность общей размерности двум. Если она не кратна двум, то нужно добавить к каждому элементу результирующей матрицы произведение последних элементов соответствующих  строки и столбца.

\section*{Вывод}

В данном разделе были рассмотрены идеи, лежащие в основе рассматриваемых алгоритмов умножения матриц - стандартного и алгоритма Винограда.