\chapter{Аналитическая часть}

В данном разделе будет приведена теория, необходимая для разработки и реализации линейного и параллельного конвейерного вариантов стандартизации массива.

\section{Конвейерная обработка данных}

Конвейеризация (или конвейерная обработка) в общем случае основана на разделении подлежащей исполнению функции на более мелкие части, называемые ступенями, и выделении для каждой из них отдельного блока аппаратуры. Так обработку любой машинной команды можно разделить на несколько этапов (несколько ступеней), организовав передачу данных от одного этапа к следующему. 

Конвейерную обработку можно использовать для совмещения этапов выполнения разных команд. Производительность при этом возрастает благодаря тому, что одновременно на различных ступенях конвейера выполняются несколько команд ~\cite{second_article}.


\section{Стандартизация данных}

В широком смысле стандартизация данных представляет собой этап их предобработки с целью приведения к определённому формату и представлению \cite{third_article}.

 Стандартизация приводит все исходные значения набора данных, независимо от их начальных распределений и единиц измерения, к набору значений из распределения с нулевым средним и стандартным отклонением, равным 1. В результате формируется так называемая стандартизированная шкала, которая определяет место каждого значения в наборе данных. Значения стандартизированной шкалы определяются по формуле (\ref{eq:ref1})
 
 \begin{equation}
 	z_i = \frac{x_i - X_{mean}}{D_x},\label{eq:ref1}
 \end{equation}

где $x_i$ — исходное значение признака, $X_{mean}$ =  $\frac{sum(x)}{count(x)}$ и $D_x$ = $\sqrt{\frac{sum((x - X_{mean})^2)}{count(x)}}$ — среднее значение и стандартное отклонение признака, оцененные по набору данных \cite{fourth_article}.




\section{Вывод из аналитической части}
В данном разделе были рассмотрены идеи и материалы, необходимые для разработки и реализации линейного и параллельного конвейерного вариантов стандартизации массива.

