\chapter{Конструкторская часть}

Рекуррентные формулы, рассмотренные в предыдущем разделе, позволяют находить расстояния Левенштейна и Дамерау-Левенштейна. Однако при разработке алгоритмов, решающих эти задачи, можно использовать различные подходы (циклы, рекурсия с кешированием, рекурсия без кеширования), которые будут рассмотрены в данном разделе.

\section{Алгоритмы поиска расстояния Левенштейна}

На рисунке \ref{img:l_matrix} приведена схема итеративного алгоритма поиска расстояния Левенштейна с заполнением матрицы расстояний.

\img{220mm}{l_matrix}{Схема итеративного алгоритма поиска расстояния Левенштейна с заполнением матрицы расстояний}

На рисунке \ref{img:l_recursion_classic} приведена схема рекурсивного алгоритма поиска расстояния Левенштейна без кеширования.

\img{180mm}{l_recursion_classic}{Схема рекурсивного алгоритма поиска расстояния Левенштейна без кеширования}

На рисунке \ref{img:l_recursion_matrix} приведена схема рекурсивного алгоритма поиска расстояния Левенштейна c кешированием.

\img{220mm}{l_recursion_matrix}{Схема рекурсивного алгоритма поиска расстояния Левенштейна с кешированием}

\section{Алгоритм поиска расстояния Дамерау - Левенштейна}

На рисунке \ref{img:Dl_recursion} приведена схема рекурсивного алгоритма поиска расстояния Дамерау-Левенштейна без кеширования.

\img{220mm}{Dl_recursion}{Схема рекурсивного алгоритма поиска расстояния Дамерау-Левенштейна  без кеширования}



\section*{Вывод}

На основе теоретических данных, полученных из аналитического раздела, были разработаны схемы алгоритмов, позволяющих с помощью различных подходов находить расстояния Левенштейна и Дамерау-Левенштейна.


