\chapter*{Введение}
\addcontentsline{toc}{chapter}{Введение}

Муравьиные алгоритмы представляют собой новый перспективный метод решения задач оптимизации, в основе которого лежит моделирование поведения колонии муравьев. Колония представляет собой систему с очень простыми правилами автономного поведения особей. Однако, несмотря на примитивность поведения каждого отдельного муравья, поведение всей колонии оказывается достаточно разумным. Эти принципы проверены временем — удачная адаптация к окружающему миру на протяжении миллионов лет означает, что природа выработала очень удачный механизм поведения ~\cite{first_article}. 


Целью данной работы является реализация муравьиного алгоритма для решения задачи коммивояжера и приобретение навыков параметризации алгоритмов.


В рамках выполнения работы необходимо решить следующие задачи: 
\begin{enumerate}[label={\arabic*)}]
	\item реализовать алгоритм полного перебора для решения задачи коммивояжера;
	\item изучить и реализовать муравьиный алгоритм для решения задачи коммивояжера;
	\item провести параметризацию муравьиного алгоритма на трех классах данных и подобрать оптимальные параметры;
	\item провести сравнительный анализ трудоемкостей реализаций.
\end{enumerate}
