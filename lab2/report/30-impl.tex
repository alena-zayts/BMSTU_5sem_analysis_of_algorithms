\chapter{Технологическая часть}

\section{Требования к ПО}

На вход программе подаются две матрицы, а на выходе должно быть получено искомое произведение матриц, посчитанное с помощью каждого реализованного алгоритма: стандартный, Винограда и оптимизированный Винограда. Также необходимо вывести затраченное каждым алгоритмом процессорное время.

\section{Выбор средств реализации}

В качестве языка программирования для реализации данной лабораторной работы был выбран язык Python  \cite{PythonBook}. Он позволяет быстро реализовывать различные алгоритмы без выделения большого времени на проектирование сруктуры программы и выбор типов данных. 

Кроме того, в Python есть библиотека time, которая предоставляет функцию process\_time для замера процессорного времени \cite{process_time_text}.

В качестве среды разработки выбран PyCharm. Он является кросс-платформенным, а также предоставляет удобный и функциональнаый отладчик и средства для рефакторинга кода, что позволяет быстро находить и исправлять ошибки \cite{pycharm}.

\section{Листинги кода}

В листингах \ref{standart} - \ref{vin_opt} представлены реализации рассматриваемых алгоритмов.

\begin{lstlisting}[caption=Стандартный алгоритм умножения матриц,
	label={standart}]
def standart_mult(a, b, c, m, n, q):
	for i in range(m):
		for j in range(q):
			for k in range(n):
				c[i][j] = c[i][j] + a[i][k] * b[k][j]
	return c
\end{lstlisting}


\begin{lstlisting}[caption=Алгоритм умножения матриц Винограда,
	label={vin}]
def vinograd_usual_mult(a, b, c, m, n, q):
	mulh = [0 for i in range(m)]
	for i in range(m):
		for k in range(n//2):
			mulh[i] = mulh[i] + a[i][k * 2] * a[i][k * 2 + 1]
	
	mulv = [0 for i in range(q)]
	for i in range(q):
		for k in range(n//2):
			mulv[i] = mulv[i] + b[k * 2][i] * b[k * 2 + 1][i]
	
	for i in range(m):
		for j in range(q):
			c[i][j] = -mulh[i] - mulv[j]
			for k in range(n//2):
				c[i][j] = c[i][j] + (a[i][k * 2] + b[2 * k + 1][j]) * (a[i][2 * k + 1] + b[2 * k][j])
	
	if n % 2:
		for i in range(m):
			for j in range(q):
				c[i][j] = c[i][j] + a[i][n - 1] * b[n - 1][j]
	
	return c
\end{lstlisting}

\begin{lstlisting}[caption=Оптимизированный алгоритм умножения матриц Винограда,
	label={vin_opt}]
def vinograd_optimized_mult(a, b, c, m, n, q):
	n -= 1
	mulh = [0 for i in range(m)]
	for i in range(m):
		for k in range(0, n, 2):
			mulh[i] -= a[i][k] * a[i][k + 1]
	
	mulv = [0 for i in range(q)]
	for i in range(q):
		for k in range(0, n, 2):
			mulv[i] += b[k][i] * b[k + 1][i]
	
	for i in range(m):
		for j in range(q):
			c[i][j] = mulh[i] - mulv[j]
			for k in range(0, n, 2):
				c[i][j] += ((a[i][k] + b[k + 1][j]) * (a[i][k + 1] + b[k][j]))
	
	if not (n % 2):
		for i in range(m):
			for j in range(q):
				c[i][j] += a[i][n] * b[n][j]
	
	return c
\end{lstlisting}



\section{Тестирование}
В таблице ~\ref{tabular:test_rec} приведены функциональные тесты для алгоритмов умножения матриц: стандартного, Винограда и оптимизированного Винограда. Все тесты пройдены успешно каждым алгоритмом.

\begin{table}[h!]
	\begin{center}
		\begin{tabular}{c@{\hspace{7mm}}c@{\hspace{7mm}}c@{\hspace{7mm}}c@{\hspace{7mm}}c@{\hspace{7mm}}c@{\hspace{7mm}}}
			\hline
			Матрица A & Матрица B &Ожидаемый результат C \\ \hline
			\vspace{8mm}
			$\begin{pmatrix}
				*
			\end{pmatrix}$ &
			$\begin{pmatrix}
				*
			\end{pmatrix}$ &
			$\begin{pmatrix}
				*
			\end{pmatrix}$ \\
			\vspace{2mm}
			\vspace{2mm}
			$\begin{pmatrix}
				2
			\end{pmatrix}$ &
			$\begin{pmatrix}
				2
			\end{pmatrix}$ &
			$\begin{pmatrix}
				4
			\end{pmatrix}$ \\
			\vspace{2mm}
			\vspace{2mm}
			$\begin{pmatrix}
				1 & 1\\
				1 & -1\\
				2 & 2
			\end{pmatrix}$ &
			$\begin{pmatrix}
				0 & -1 & 1 & 2\\
				0 & 1 & 1 & 3
			\end{pmatrix}$ &
			$\begin{pmatrix}
				0 & 0 & 2 & 5\\
				0 & -2 & 0 & -1\\
				0 & 0 & 4 & 10
			\end{pmatrix}$ \\
			\vspace{2mm}
			\vspace{2mm}
			$\begin{pmatrix}
				1 & 1 & 1\\
				1 & 1 & 1\\
				1 & 1 & 1
			\end{pmatrix}$ &
			$\begin{pmatrix}
				1 & 2 & 3\\
				0 & 0 & 0\\
				1 & 1 & 1
			\end{pmatrix}$ &
			$\begin{pmatrix}
				2 & 3 & 4\\
				2 & 3 & 4\\
				2 & 3 & 4
			\end{pmatrix}$ \\
			\vspace{2mm}
			\vspace{2mm}
			$\begin{pmatrix}
				1 & 1\\
				2 & 2
			\end{pmatrix}$ &
			$\begin{pmatrix}
				1
			\end{pmatrix}$ &
			Не могут быть перемножены\\
		\end{tabular}
	\end{center}
	\caption{\label{tabular:test_rec} Тестирование функций}
\end{table}


\section*{Вывод}

Был производен выбор средств реализации, приведены требования к ПО, реализованы и протестированы алгоритмы умножения матриц.
