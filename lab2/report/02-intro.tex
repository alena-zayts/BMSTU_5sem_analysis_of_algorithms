\chapter*{Введение}
\addcontentsline{toc}{chapter}{Введение}

Целью данной лабораторной работы является изучение способов оптимизации алгоритмов на примере алгоритмов умножения матриц.

Основное значение термин «матрица» имеет в математике. Матрица — математический объект, записываемый в виде прямоугольной таблицы элементов кольца или поля (например, целых или комплексных чисел), которая представляет собой совокупность строк и столбцов, на пересечении которых находятся её элементы. Количество строк и столбцов матрицы задают размер матрицы \cite{matrix}.

Матрицы широко применяются в математике для компактной записи систем линейных алгебраических или дифференциальных уравнений. В этом случае, количество строк матрицы соответствует числу уравнений, а количество столбцов — количеству неизвестных. В результате решение систем линейных уравнений сводится к операциям над матрицами, в том числе - умножению.

Умножение матриц A и B — это операция вычисления матрицы C, элементы которой равны сумме произведений элементов в соответствующей строке первого множителя и столбце второго \cite{matrix2}.


В рамках выполнения работы необходимо решить следующие задачи: 
\begin{enumerate}[label={\arabic*)}]
	\item изучить алгоритмы умножения матриц;
	\item разработать и реализовать 3 алгоритма умножения матриц: стандартный, Винограда и Винограда с оптимизациями;
	\item оценить трудоемкость реализаций алгоритмов;
	\item провести сравнительный анализ процессорного времени выполнения реализаций алгоритмов.
\end{enumerate}
